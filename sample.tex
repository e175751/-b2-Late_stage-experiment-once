\documentclass[a4paper,11pt,titlepage]{jarticle}
\usepackage[dvipdfmx]{graphicx}
\usepackage{listings}
\usepackage{amsmath}
\usepackage{fancybox,ascmac}
\usepackage{url}

\title{知能情報実験II : 第二回レポート}
\author{175751C 宮城孝明}
\date{\today}

\begin{document}
\maketitle
\tableofcontents
\clearpage
\section{実験目的}
 教育用ワンボートコンピュータKUE-CHIP2を用いて例題プログラムを実行する
 ことにより,KUE-CHIP2 の操作方法を習得する.また,簡単な機械語(マシン語)
 プログラミングの作成を通して,KUE-CHIP2の命令セットおよび機械語プログラム
 の構造を理解することを目的とする.
 \section{実験概要}
 コンピューターアーキテクチャについて学べるKUE-CHIP2を用いて,プログ
 ラミングを入力し,そして,出力の確認をした.初めに,簡単な説明を受けマニュア
 ルの1章目を見ながら,KUE-CHIP2の使い方を確認した.そして,マニュアルの3
 章目内容に従い,例題プログラムを実行させた.その次に,(a)~(r)までの操作を
 KUE-CHIP2で実際に行った.次に,先の例題プログラムの少し改変した課題を取
 り組む.最後に余裕のある人が,追加課題を行った. これにより,パソコンの内部
 の動きがよりわかりやくなった.さらに,入力の際は手動で入力することで,アセ
 ンブラ言語から機械語(マシン語)への変換作業を行った.変換作業の時には,先に
 もらった対応表での確認や,値をロードしストアするときに,データ領域のどこ
 にあるのかを確認に気をつける.
 \section{実験結果}
 \begin{enumerate}
  \item 本実験の結果に関しては,報告を省略する.\par
  \item 本実験の結果に関しては,報告を省略する.\par
  \item 本実験の結果に関しては,報告を省略する.\par
  \item 実験の報告をする.\par
     図1のコード見て欲しい.まずは,データ領域(01H)の値をIXに代入する.
    その次に,ACCとACCの排他的論理和をとる.このことにより,ACCの数値は0になる.ACCと(00H)を足し算し,
    その結果をACCに代入する.その次の階層では,最初に代入したIXを1ずつ減らしていく.条件分岐ではIXが0になるまで繰り返す.
    戻るところをACCと(00H)の足し算にする.そうすることで,IXの中身が1ずつ経るごとにACCは
    (00H)の値が足されていく.これにより,IXが0になった時には,ACCの値は(00H)の値と(01H)の値の
    乗算結果が格納されている.最後にACCの値をデータ領域(02H)に格納する.結果は,(00H)は4,(01H)は,
    5が代入されているため,(02H)には20が格納される.

\begin{figure}[htbp]
  \centering
  \includegraphics[width=38mm]{sample1.png}
  \label{実験(4)のフローチャート}\\
  \caption{実験(4)のフローチャート}
\end{figure}

  \lstinputlisting[numbers=left,breaklines=true,basicstyle=\ttfamily\footnotesize,
   frame=sing,caption=実験(4)のコード,label=1]{Sample.s}
  \item 実験の報告をする.\par
     図2のフローチャートを見て欲しい.追加課題では,1から最大値までの足し算することである.
    そのためまずは,データ領域(00H)の値(最大値)をIXに代入する.
    その次は,ACCとACCの排他的論理和をとる.これのことより,ACCの値が0になる.
    次の階層では,ACCとIXの足し算をし,その合計をACCに代入する.そして,IXの値を1ずつ減らしていく.
    条件分岐では,ACCとIXの足し算のところに戻るようにする.それをIXが0になるまで繰り返す.
    そして,IXが0になった時には,ACCの値は1から最大値までの和となっている.最後にACCの値を
    データ領域(01H)に格納する.結果は,(00H)には20が格納されており,1になるまで処理を繰り返すため
    (01H)には210が格納されている.
    \begin{figure}[htbp]
      \centering
      \includegraphics[width=35mm]{sample2.png}
      \label{実験(5)のフローチャート}\\
      \caption{実験(5)のフローチャート}
    \end{figure}
 \lstinputlisting[numbers=left,breaklines=true,basicstyle=\ttfamily\footnotesize,
  frame=sing,caption=実験(5)のコード,label=2]{Sample1.s}
 \end{enumerate}
 \section{考察}
 \begin{enumerate}
  \item 実験(1),(2),(3)について \par
      今回の実験では,メモリやカウンタ,レジスタなどの書き込みを変更する際のselの動作に
    改善案がある.現状だと,パソコンとは勝手が異なるため,書き込みの際に誤った場所にデータを
    格納したら確認する時に発見に遅れ,時間を掛けてしまい,学習者の意欲を失わせる可能性がある.
    初見でありプログラミングに苦手意識がある生徒であれば尚更,そのようなことになってしまう.
    改善案としては,出力画面に格納先の情報を載せる.その他に,selのレバー近辺に格納先をランプ
    のような物で伝えるなどすれば良いと思われる.
  \item 実験(4),(5)について\par
     C言語やjavaとアセンブラ言語には,一度機械語に翻訳しないといけないという同一の点がある.\par
   相違点としては,アセンブラは単純なプログラムを書くことはできるが,プログラムが複雑になれば
    行数も増え,困難になっていく.単純なため,書くときのパターンもそれほど多くはない.しかし,
    変数は型などを気にしなくて良い.\par
     C言語やjavaは,プログラムが複雑になっても対応できる言語である.書くときのパターンも豊富
    である.しかし,文字列や数字、文字一つの単体によって,変数の型が変わるという点がる.\par
 \end{enumerate}
 \section{調査課題}
 \begin{enumerate}
  \item コンピュータの主要構成要素に関して
  \par
     入力装置には,キーボードと音声入力,スキャナ,マウス,バーコード読み取り機など他に多くある.\par
     補助記憶装置には,ハードディスク装置,SSD,SDカード,CD-R/RWドライブ,ブルーレイドライブ,
    DVD-R/RWドライブなどがあげられる.\par
     出力装置には,ディスプレイ,プリンタ,有機ELディスプレイ,プラズマディスプレイなどがあげ
    られる.\par
     私のもつCPUには,intel Core i5が使われており,インテルが生産し,i5-6360Uの型番になっている.
    AMDが生産している.Ryzen Threadripper 1950Xが型番となっている.さらに,もう一つは
    TSMCは生産している.A11 Bionicが型番になっている.\par
      intel Core i5は,一般層向けの製品である.ゲームや動画編集はできるが,
    i7にはまだまだ劣る.AMDは,intelよりも安価で性能の違いもそれほどない.
    A11 Bionicは,iPhpne x/8/8plasに使用されており,高性能のCPUコア「Monsoon」
    と高効率のCPUコア「Mistral」を使っている.これをうまく交互に使い分けてCPUの
    性能を向上させる.
  \item
     多種多様な記憶装置が用いられる理由として,現在のノイマン型つまり,逐次処理を行う
    PCである.プログラム内蔵方式のため,データは一度バスを通してメモリに格納しなければならない.
    しかし,プロセッサと記憶装置との間のアクセスが遅ければフォン・ノイマン・ボトルネック
    になる.これを解決するために,記憶装置の階層化が必要である.プロセッサと記憶装置の間にキャッシュ
    メモリ,記憶装置とハードディスクの間にディスクキャッシュを置くことで,相互の速度差を埋める
    ことができる.搭載箇所としては,はじめにプロセッサのレジスタが一番速い.次に,キャッシュメモリは
    記憶装置をレジスタの間にあり,アクセス速度速くする役割である.その次が,記憶装置である.
    プログラムに必要なデータを格納する役割である.ディスクキャッシュは,記憶装置とハードディスク
    の間にあり,相互のアクセス速度を埋めてくれる役割である.最後に,ハードディスクはデータの保存を
    してくれる役割である.レジスタ,キャッシュメモリの順に容量は少ないし,高価であるが,アクセス速度
    がとても速い.
 \end{enumerate}
 \section{感想}
  今回の実験は,普段使用するPCと同じ機能を持つが値の格納をデータ領域の何番値にするのか
  作成者本人が行わないといけないことが,とても新鮮で苦労した.さらに,実行中のエラー報告も
  ないため,格納先で値がちゃんと合っているのかの確認をしなければ,エラーがあるのかどうかも
  わからないという点は,非常に苦労した.
 \section{引用文献}
 \begin{thebibliography}{99}
  \bibitem{wikipedia}wikipedia: https://ja.wikipedia.org/wiki/Intel\_Core\_i5
  \bibitem{pcinformation}pcinformation: http://pcinformation.info/cpu/
  corei9\-corei7\-corei5\-corei3\-difference.html
  \bibitem{chimolog}chimolog: https://chimolog.co/bto\-soc\-a11/
  \bibitem{weblio}weblio: https://www.weblio.jp/wkpja/content/フォン・ノイマン・ボトルネック\_フォン・ノイマン・ボトルネックの概要
  \bibitem{hikaku}hikaku: https://btopcs.jp/hikaku/hikaku\_cpu/
 \end{thebibliography}

\end{document}
